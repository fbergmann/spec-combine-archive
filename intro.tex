% -*- TeX-master: "main"; fill-column: 72 -*- 


\section{Document conventions} \label{conventions} 

Following the precedent set by other COMBINE specification
documents, we use UML~1.0 (Unified Modeling Language; 
\citealt{eriksson:1998,oestereich:1999}) class diagram notation to 
define the constructs provided by this package. 

We also use the following typographical conventions to distinguish the 
names of objects and data types from other entities; these conventions 
are identical to the conventions used in the other COMBINE 
specification documents: 

\begin{description} 

\item \abstractclass{AbstractClass}: Abstract classes are classes that 
are never instantiated directly, but rather serve as parents of other 
object classes. Their names begin with a capital letter and they are 
printed in a slanted, bold, sans-serif typeface. In electronic document 
formats, the class names defined within this document are also 
hyperlinked to their definitions; clicking on these items will, given 
appropriate software, switch the view to the section in this document 
containing the definition of that class. 

\item \class{Class}: Names of ordinary (concrete) classes begin with a 
capital letter and are printed in an upright, bold, sans-serif typeface. 
In electronic document formats, the class names are also hyperlinked to 
their definitions in this specification document. 

\item \token{SomeThing}, \token{otherThing}: Attributes of classes, data 
type names, literal XML, and generally all tokens \emph{other} than 
UML class names, are printed in an upright typewriter typeface. 
Primitive types defined begin with a capital letter; the COMBINE archive
also makes use of primitive types defined by XML 
Schema~1.0~\citep{biron:2000,fallside:2000,thompson:2000}, but 
unfortunately, XML~Schema does not follow any capitalization convention 
and primitive types drawn from the XML~Schema language may or may not 
start with a capital letter. 

\end{description} 

For other matters involving the use of UML and XML, we follow the 
conventions used in other COMBINE specification documents. 

\section{Background}  \label{background} 

\subsection{ Motivation and Background}

Computational modeling is an increasingly interdisciplinary field, 
different aspects come together that need to be stored in a cohesive 
unit. When exchanging a model, it increasingly becomes an issue that not 
all relevant files are exchanged along with it. Many different 
approaches have been taken to solve this issue, such as folder based 
project structures, or special versions of version control systems. 
Unfortunately these approaches are not as easy to support for tool 
authors as a single file based solution would. 

This specification describes the "COMBINE archive" format. A 
\textit{COMBINE archive} is a single file containing the various 
documents (and in the future, references to documents), necessary for 
the description of a model and all associated data and procedures. This 
includes for instance, but not limited to, simulation experiment 
descriptions in SED-ML, all models needed to run the simulations in SBML 
and their graphical representations in SBGN-ML. 

The COMBINE archive aims to augment these efforts by standardizing a 
meta file format (or manifest) that describes what files all belong to 
computational models, as well as a description format and a convention 
for bundling the information together. 

Details of earlier independent proposals are provided in 
\ref{background}. 

