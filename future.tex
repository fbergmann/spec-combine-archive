% -*- TeX-master: "main"; fill-column: 72 -*-

\section{Future development}
\label{future}
In this section we highlight some open issues not addressed in this 
version of the COMBINE archive specification. 


\subsection{Linking to external documents}
It was often discussed to also allow the \token{location} elements of a 
\Content element point to an external document. However, in this first 
version we restrict them to local files, so as to make it easier to 
adopt in software tools. That way tools could focus on the primary use 
case of bundling up several local resources, rather than worry about 
retrieving information from online resources that may not always be 
available or may be more complex.

\subsection{Cross References between entries}
At HARMONY 2013, we spend some time discussing whether cross references 
between the individual entries in the archive ought to be in this first 
version of the specification. It was decided to leave the cross 
referencing to the individual standards for now, rather to impose them 
ad-hoc. 

\subsection{Alternative versions of the archive metadata}
It was suggested to allow different versions of the archive metadata 
format. The manifest already provides a way for referencing alternate 
versions, all that would need to be changed, would be the format 
identifier to point to a different namespace rather than: 


\url{http://identifiers.org/combine.specifications/omex-metadata}

However, as of the time of this writing no such format was proposed.

\subsection{Convergence with other archive formats}
Recently, the Workflow4Ever project (Wf4ever) funded by the European Union 
developed the Research Object Bundle \citep{ro2013}, focusing on the preservation
of scientific experiments in data-intensive science. The structure of the 
Research Object Bundle is close to the COMBINE archive. Based on Adobe 
Universal Container Format \citep{ucf}, it is also a Zip file containing 
a manifest and metadata. Discussions have started with the latter project. 
One way of starting convergence would be to share metadata vocabularies and format. 


Further from scientific research, several industry standards are being largely 
used for building document archives based on Zip files. The main ones belong to 
two large families. The aforementioned Universal Container Format bears strong 
similarities with the Open Document Format \citep{ODF} developed by the Organization 
for the Advancement of Structured Information Standards (OASIS), and the EPUB 
Open Container Format  \citep{OCF} developed by the International Digital Publication 
Forum (IDPF). As for the format described here, those archives contain a manifest 
file that lists and identifies the contents of the archive and a metadata file, 
either in xml or rdf. In parallel, Microsoft developed the Open Packaging Convention  \citep{OPC} 
used in its office suite and some other software such as MathWorks' Simulink. Those 
packages also list their contents in an xml file, and can carry metadata. 


Those industry standards are relatively complex and rigid. In this initial iteration 
of the specification, the community opted for simplicity and flexibility, in order 
to get traction. The main differences between the COMBINE archive and the formats 
discussed above lie in the structure of the archive and the format of the file 
describing the content. Since the COMBINE archive is in general less specified, 
future convergences will be easy to implement once agreed upon.
