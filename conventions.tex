\section{Document conventions} \label{conventions} 

Following the precedent set by other COMBINE specification
documents, we use UML~1.0 (Unified Modeling Language; 
\citet{eriksson:1998,oestereich:1999}) class diagram notation to 
define the constructs provided by this package. 

We also use the following typographical conventions to distinguish the 
names of objects and data types from other entities: 

\begin{description} 

\item \class{Class}: Names of ordinary (concrete) classes begin with a 
capital letter and are printed in an upright, bold, sans-serif typeface. 
In electronic document formats, the class names are also hyperlinked to 
their definitions in this specification document. 

\item \token{SomeThing}, \token{otherThing}: Attributes of classes, data 
type names, literal XML, and generally all tokens \emph{other} than 
UML class names, are printed in an upright typewriter typeface. 
Primitive types defined begin with a capital letter; the COMBINE archive
also makes use of primitive types defined by XML 
Schema~1.0~\citep{biron:2000,fallside:2000,thompson:2000}, but 
unfortunately, XML~Schema does not follow any capitalization convention 
and primitive types drawn from the XML~Schema language may or may not 
start with a capital letter. 

\end{description} 
